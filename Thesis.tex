%
% Document template suitable for use as a latex master-file for masters
% thesis in University of Turku Department of Information Technology. 
% Relies on itpackage.sty for additional definitions.
%
% Sami Nuuttila (samnuutt@utu.fi) 
%
% Last mod 2.9.2015:
% 
% Why:
%  - No need for anyone to invent the wheel again. You can of course do that
%    if you wish - TeX'll even let you invent many different kinds of wheels
%  - That said, if you come up with a great new wheel I'd like to hear about 
%    it - it might even end up being used here 
% 
% Features:
%  - Proper page sizes as required by university guide for students:
%      - proper font sizes as well as linespacings
%      - proper size of margins
%  - Generic title page:
%      - \gentitle
%  - Generic abstract page(s):
%      - \begin{itabstract}{Keywords}
%          abstract text
%        \end{itabstract}
%      - \begin{ittiivis}...\end{ittiivis} provides finnish version
%         - ittiivis defaults to finnish so no need to issue 
%           \selectlanguage{finnish}
%      - total number of pages as well as total number of pages in appendices
%        are automagically handled
%  - Entry environment:
%      - \begin{entry}[widest label]
%          \item[1st label text] ...
%          \item[2nd label text] ...
%        \end{entry}
%      - the actual items are aligned to suit the widest label, which is
%        given as an argument to the environment
%  - Use of specific latex packages to ease in formatting the thesis:
%      - format table of contents to have bibliography shown as references
%        as well as other fixes           (tocbibind)
%      - enhanced verbatim handling       (sverb)
%      - source code inclusion            (listings)
%      - handling of headers and footers  (fancyhdr)
%
%      - consultation of the manuals of these packages is strongly
%        encouraged 
%
% Assumptions:
%  - itpackage.sty file is available 
%  - each chapter is as a separate file which is read in with e.g. \input
%   
% Miscellaneous:
%  - comments are welcome
%  - should a required package be missing see http://www.ctan.org/ 
%  - http://www.ctan.org/tex-archive/info/lshort/english/lshort.pdf
%asd
% Later modifications:
%  - 22.9.2016, 1.10.2017:Johannes Holvitie <jjholv@utu.fi>, additions
%    marked "JH:". Reason: modified for use in tex.soft.utu.fi
%
%%%%%%%%%%%%%%%%%%%%%%%%%%%%%%%%%%%%%%%%%%%%%%%%%%%%%%%%%%%%%%%%%%%%%%%%%%%
%
% load all required packages
%
%%%%%%%%%%%%%%%%%%%%%%%%%%%%%%%%%%%%%%%%%%%%%%%%%%%%%%%%%%%%%%%%%%%%%%%%%%%

% document is based on report class
\documentclass[a4paper,12pt]{report}

% load ams-packages for maths
\usepackage{amssymb,amsthm,amsmath}

% load babel-package for automatic hyphenation
\usepackage[english]{babel}
% JH: modified latin to UTF-8 encoding cues to make Scandinavian characters works
\usepackage[utf8]{inputenc}

% load graphicx package
%   - automagically select proper parameters depending on whether
%     we're running pdflatex or latex
%   - specify \includegraphics{file} without the file extension
%     (.eps /.pdf (/ .png / .jpeg / ...)), tex should select the proper file
%
\usepackage{ifpdf}
\usepackage{graphicx}
%% !!! NOTE: if you have ancient LaTeX distribution then you might need to
%% use the following instead
%% \newif\ifpdf
%% \ifx\pdfoutput\undefined
%%   \pdffalse
%% \else
%%   \pdfoutput=1
%%   \pdftrue
%% \fi
%% % if graphicx complains about option clash remove the [pdftex] option
%% \ifpdf
%%   \usepackage[pdftex]{graphicx}
%% \else
%%   \usepackage[dvips]{graphicx}
%% \fi


% load tocbibind package 
%   - do not include table of contents in itself
%   - convert the name of bibliography to references
\usepackage[nottoc]{tocbibind}

% load sverb package
%   - enhanced handling of verbatim stuff; listing environment
\usepackage{sverb}

% load listings package
%   - handle inclusion of source code
\usepackage{listings}

% load fancyheaders package
%   - the actual headers and footers are set later
\usepackage{fancyhdr}

% load itpackage 
%   - additional defines and stuff
\usepackage{thesis/itpackage}

% uncomment the following snippet to get rid of Luku/Chapter text at the
% beginning of each Chapter... 
%\makeatletter
%\renewcommand{\@chapapp}{\relax}
%\renewcommand{\@makechapterhead}[1]{%
%  \vspace*{50\p@}%
%  {\parindent \z@ \raggedright \normalfont
%    \ifnum \c@secnumdepth >\m@ne
%        \huge\bfseries \@chapapp\space \thechapter\space\space
%    \fi
%    \interlinepenalty\@M
%    \Huge \bfseries #1\par\nobreak
%    \vskip 40\p@
%  }}
%\makeatother
\begin{document}

% Select language here by removing comment from one of the two rows below
%\selectlanguage{finnish}\fintrue
\selectlanguage{english}\finfalse

% Set some names based on selected language; no modification required
\iffin
\settocbibname{Lähdeluettelo}
\renewcommand{\appname}{Liitteet}
\else
\settocbibname{References}
\renewcommand{\appname}{Appendices}
\fi

% Fill in your information below
\workinfo{Jani Anttonen}
{Proof of Proximity Using a Verifiable Delay Function}
{Jaakko Järvi}
{Second Supervisor}
{2020}
{November}
{Kuukausi}

% Set the type of your thesis (Diplomityö, TkK -tutkielma, etc.) and
% laboratory or field of study below
\worktype{Type of thesis}{M.Sc.} 
\deptinfo{Laboratory Name}{Tulevaisuuden teknologioiden laitos}

% Generate the title page 
\gentitle

% Include the abstract
% !TeX root = ../index.tex
\begin{itabstract}{list, of, keywords}
	Tarkempia ohjeita tiivistelmäsivun laadintaan läytyy opiskelijan
	yleisoppaasta, josta alla lyhyt katkelma.
	
	Bibliografisten tietojen jälkeen kirjoitetaan varsinainen tiivistelmä.
	Sen on oletettava, että lukijalla on yleiset tiedot aiheesta.
	Tiivistelmän tulee olla ymmärrettävissä ilman tarvetta perehtyä koko
	tutkielmaan. Se on kirjoitettava täydellisinä virkkeinä,
	väliotsakeluettelona. On käytettävä vakiintuneita termejä. Viittauksia
	ja lainauksia tiivistelmään ei saa sisällyttää, eikä myäskään tietoja
	tai väitteitä, jotka eivät sisälly itse tutkimukseen. Tiivistelmän on
	oltava mahdollisimman ytimekäs n. 120 -- 250 sanan pituinen itsenäinen
	kokonaisuus, joka mahtuu ykkäsvälillä kirjoitettuna vaivatta
	tiivistelmäsivulle. Tiivistelmässä tulisi ilmetä mm.  tutkielman aihe
	tutkimuksen kohde, populaatio, alue ja tarkoitus käytetyt
	tutkimusmenetelmät (mikäli tutkimus on luonteeltaan teoreettinen ja
	tiettyyn kirjalliseen materiaaliin, on mainittava tärkeimmät
	lähdeteokset; mikäli on luonteeltaan empiirinen, on mainittava käytetyt
	metodit) keskeiset tutkimustulokset tulosten perusteella tehdyt
	päätelmät ja toimenpidesuositukset asiasanat
\end{itabstract}


% empty pagestyle for table of contents etc. 
%
% the redefinition of plain style works also for 1st pages of chapters,
% which is the default in report class. Just state \thispagestyle{empty}
% after \chapter{something} if you want empty style for the 1st pages. 
%
\pagestyle{empty}
\fancypagestyle{plain}{
  \fancyhf{}
  \renewcommand{\headrulewidth}{0 pt}
}

% roman numbering for table of contents etc.
\pagenumbering{roman}

% insert table of contents, list of figures, list of tables
%
% setting the counter to zero effectively removes the page number from
% the toc, lof etc. entries in the toc since there is no roman numeral
% for zero ;-) (if you want them without numbering)
%
%\setcounter{page}{0}
\tableofcontents
\clearpage
%\setcounter{page}{0}
%\listoffigures 
%\clearpage
%\setcounter{page}{0}
%\listoftables

% possibly insert 'list of acronyms'
%
%   - create a chapter called List Of Acronyms (or whatever), which
%     should contain all your acronym definitions, e.g. 
%     \chapter{List Of Acronyms} 
%   - the secnumdepth trickery is needed because acronyms are as a
%     standard chapter and we are faking '\listofacronyms'
%
%\setcounter{secnumdepth}{-1}
%\input{your acronym chapter's file name}
%\setcounter{secnumdepth}{2}

% setup page numbering, page counter, etc.
\startpages

%%%%%%%%%%%%%%%%%%%%%%%%%%%%%%%%%%%%%%%%%%%%%%%%%%%%%%%%%%%%%%%%%%%%%%%%%%%
%
% Thesis starts here. Create a new tex file for each chapter and input it below. You may encounter errors if you use å, ä, ö or <space> characters in referred names.
%
% Good luck!
%
%%%%%%%%%%%%%%%%%%%%%%%%%%%%%%%%%%%%%%%%%%%%%%%%%%%%%%%%%%%%%%%%%%%%%%%%%%%

% !TeX root = ./Thesis.tex
\chapter{Introduction}
\label{Introduction}

Computer applications, whether they are on the public Internet or on a private network, commonly use the client-server model of communication. In this model, the client application sends requests to the server, and the server responds. Applications are today, however, increasingly moving towards a distributed, peer-to-peer (P2P) networked model, where every peer, whether it is an entire computer or merely a process running on one, serves equally as the both sides of the client-server model. Many uses of P2P are not visible to the end user, because in many cases, P2P technologies serve to optimize resource utilization rather than as centerpoints of applications. For example, the music subscription service Spotify's protocol has been designed to combine server and P2P streaming to decrease the load on Spotify's servers and save bandwith, and thus improve the service's scalability.~\cite{Kreitz_undated-yp} Naturally, P2P streaming comes in handy also whenever there is a short outage, as users might not experience any stoppages to the service, at least when listening to widely streamed content. 

The aspects of peer-to-peer networking that have recently gotten attention are cryptocurrency and blockchain technologies, categorized under the roof term of distributed ledger technologies. Prior to the emergence of ledger technologies, peer-to-peer systems were often seen in the public light as a technology to work around regulations and for doing lawless activities. This is partly because before their use in blockchain applications, P2P systems were popularized for their use in file sharing applications like Bittorrent. Obviously, this reputation was not earned for nothing, but peer-to-peer networking's use in torrents and other file sharing networks also demonstrates its potential in content delivery networks, or CDNs for short. Peer-to-peer networks are not without problems. One persistent source of problems is routing and peer discovery. Since the users of a P2P network do not necessarily connect to a centralized and optimized server infrastructure with vast bandwidth and computation resources, the individual connections between peers can be ephemeral, and routes between peers that are not directly connected are usually random. This results in some routes between peers to be inoptimal, which could result in bad performance and even false states in distributed ledgers, where slow data propagation in a P2P publish-subscribe network could mean inability to participate in the consensus due to latency. % AND THEN WHAT yes, describe the problem

Blockchain networks need a way of synchronizing the latest state of the blockchain globally between a number of peers on a P2P network. Since the blockchain data model is sequential and all recorded history is immutable, an algorithm is needed to reach total synchronization and agreement between peers. This kind of an algorithm is called a consensus algorithm. Consensus problems are not unique to blockchains, but are also relevant in distributed databases. Blockchains have raised new issues in consensus algorithms that have sparked an ongoing development effort for new kinds of consensus algorithms.

Proof of Work is the most used consensus algorithm in public blockchains today, including in Bitcoin, where it was first described in 2008.~\cite{Nakamoto2019-ax} In short, in Proof of Work all participants try to guess a correct answer to a cryptographic puzzle as fast as they can, and a correct answer produces a new block. The new block gets sent and propagated to the network, and if enough peers agree that the answer was correct, the guessing game starts over. In Proof of Work, any latency at all results in wasted resources for all participants before they receive the latest block, as they are still trying to solve the puzzle before they know the result.

As latency is not only a physical and a computational problem in P2P networks, but rather an algorithmic one, search for alrorithmic solutions has long been a part of P2P networks research. Since latency between peers in P2P networks is a problem of the network topology, the solutions have tried to optimize the peer discovery mechanisms. This has worked to some degree, with randomized peer discovery and a high amount of concurrent connections resulting in P2P networks with good peer discoverability and resource availability, but with subpar routes between peers that are not directly connected to each other.

If a peer told you its individual latencies to other peers on its routing table, couldn't this help with constructing an optimal distributed routing table, with peers only connecting to other peers if they are actually close-by or behind a fast connection? I think that it could, but since only a certain number of connections can be kept at one time, this would increase the possibility for so-called eclipse attacks. In an eclipse attack, an attacker tries to influence the victim's connections in a way that it would only be connected to peers controlled by the attacker, effectively isolating the victim from honest peers on the network, increasing the possibility for man-in-the-middle attacks regarding the victim's sent requests or received responses. Also, a not-so-malicious peer could try to game the peer discovery algorithm in the pursuit of better connections to itself at the cost of others. Whilst this is not as severe of a problem, it's still undesirable.

Now, what if you didn't have to trust the peer who reports the latencies, and there was no way for the other peer to lie on the subject? Given this, a peer could roughly estimate the network topology and find the peers closest to it with a roughly optimal amount of queries, and the P2P network could converge towards an optimal topology with limited threat of eclipse attacks. Such a proof of latency could also be used to prove a geographical location, which could be used to battle GPS spoofing. Usually, GPS is a user-facing technology and an individual tool for pathfinding, with only the user requiring the coordinates to be correct so that they won't drive to the wrong cottage. Some applications might require some sort of a proof of a physical location, and this can be done in a centralized manner with bluetooth beacons or with something resembling COVID-19 contact tracing. In a trustless P2P setting, however, this is not as easily achievable without introducing an incentive structure or requiring the use of trusted computation modules.  


% TODO: Elaborate on eclipse attacks, add citations on the matter
Using a verifiable delay function, I propose a novel algorithm for producing a publicly verifiable proof of network latency and difference in computation resources between two participants in a peer-to-peer network. This proof can be used for dynamic routing to achieve better, faster routes between peers, and for making eclipse attacks\footnote{Eclipse attack means polluting the target's routing table restricting the target's access to the rest of the network, which opens up other attack possibilities.} harder to achieve. 

\chapter{Toisen luvun otsikko}
\label{Toinen luku}

Lisää tekstiä. Lisää tekstiä. Lisää tekstiä. Lisää tekstiä. Lisää tekstiä. Lisää tekstiä. Lisää tekstiä. Lisää tekstiä. Lisää tekstiä. Lisää tekstiä. Lisää tekstiä. Lisää tekstiä.

asfasfsafsafs
%\input{file_name_of_chapter_x}
%\input{file_name_of_chapter_y}


% insert references
%  - included unsrtf.bst provides finnish language version of unsrt
\iffin
\bibliographystyle{thesis/unsrtf}
\else
\bibliographystyle{unsrt}
\fi
\bibliography{library}

%%%%%%%%%%%%%%%%%%%%%%%%%%%%%%%%%%%%%%%%%%%%%%%%%%%%%%%%%%%%%%%%%%%%%%%%%%%
%
% Almost there....
%
%%%%%%%%%%%%%%%%%%%%%%%%%%%%%%%%%%%%%%%%%%%%%%%%%%%%%%%%%%%%%%%%%%%%%%%%%%%

% make sure pagecount is correct even if references overflow to a new page
\pagebreak\addtocounter{page}{-1}
\eofpages
\appendices

% create your appendix chapters with command \appchapter{some name} instead
% of \chapter{some name} for the automagic page counting to work
%\input{file name of appchapter xxx}
%\input{file name of appchapter yyy}
%\input{file name of appchapter zzz}
%\input{and so on}

%%%%%%%%%%%%%%%%%%%%%%%%%%%%%%%%%%%%%%%%%%%%%%%%%%%%%%%%%%%%%%%%%%%%%%%%%%%
%
% main document ends here
%
%%%%%%%%%%%%%%%%%%%%%%%%%%%%%%%%%%%%%%%%%%%%%%%%%%%%%%%%%%%%%%%%%%%%%%%%%%%
\eofapppages
\end{document}
