%
% Document template suitable for use as a latex master-file for masters
% thesis in University of Turku Department of Information Technology. 
% Relies on itpackage.sty for additional definitions.
%
% Sami Nuuttila (samnuutt@utu.fi) 
%
% Last mod 2.9.2015:
% 
% Why:
%  - No need for anyone to invent the wheel again. You can of course do that
%    if you wish - TeX'll even let you invent many different kinds of wheels
%  - That said, if you come up with a great new wheel I'd like to hear about 
%    it - it might even end up being used here 
% 
% Features:
%  - Proper page sizes as required by university guide for students:
%      - proper font sizes as well as linespacings
%      - proper size of margins
%  - Generic title page:
%      - \gentitle
%  - Generic abstract page(s):
%      - \begin{itabstract}{Keywords}
%          abstract text
%        \end{itabstract}
%      - \begin{ittiivis}...\end{ittiivis} provides finnish version
%         - ittiivis defaults to finnish so no need to issue 
%           \selectlanguage{finnish}
%      - total number of pages as well as total number of pages in appendices
%        are automagically handled
%  - Entry environment:
%      - \begin{entry}[widest label]
%          \item[1st label text] ...
%          \item[2nd label text] ...
%        \end{entry}
%      - the actual items are aligned to suit the widest label, which is
%        given as an argument to the environment
%  - Use of specific latex packages to ease in formatting the thesis:
%      - format table of contents to have bibliography shown as references
%        as well as other fixes           (tocbibind)
%      - enhanced verbatim handling       (sverb)
%      - source code inclusion            (listings)
%      - handling of headers and footers  (fancyhdr)
%
%      - consultation of the manuals of these packages is strongly
%        encouraged 
%
% Assumptions:
%  - itpackage.sty file is available 
%  - each chapter is as a separate file which is read in with e.g. \input
%   
% Miscellaneous:
%  - comments are welcome
%  - should a required package be missing see http://www.ctan.org/ 
%  - http://www.ctan.org/tex-archive/info/lshort/english/lshort.pdf
%asd
% Later modifications:
%  - 22.9.2016, 1.10.2017:Johannes Holvitie <jjholv@utu.fi>, additions
%    marked "JH:". Reason: modified for use in tex.soft.utu.fi
%
%%%%%%%%%%%%%%%%%%%%%%%%%%%%%%%%%%%%%%%%%%%%%%%%%%%%%%%%%%%%%%%%%%%%%%%%%%%
%
% load all required packages
%
%%%%%%%%%%%%%%%%%%%%%%%%%%%%%%%%%%%%%%%%%%%%%%%%%%%%%%%%%%%%%%%%%%%%%%%%%%%

% document is based on report class
\documentclass[a4paper,12pt]{report}

% load ams-packages for maths
\usepackage{amssymb,amsthm,amsmath}

% load babel-package for automatic hyphenation
\usepackage[english]{babel}
% JH: modified latin to UTF-8 encoding cues to make Scandinavian characters works
\usepackage[utf8]{inputenc}

% load graphicx package
%   - automagically select proper parameters depending on whether
%     we're running pdflatex or latex
%   - specify \includegraphics{file} without the file extension
%     (.eps /.pdf (/ .png / .jpeg / ...)), tex should select the proper file
%
\usepackage{ifpdf}
\usepackage{graphicx}
%% !!! NOTE: if you have ancient LaTeX distribution then you might need to
%% use the following instead
%% \newif\ifpdf
%% \ifx\pdfoutput\undefined
%%   \pdffalse
%% \else
%%   \pdfoutput=1
%%   \pdftrue
%% \fi
%% % if graphicx complains about option clash remove the [pdftex] option
%% \ifpdf
%%   \usepackage[pdftex]{graphicx}
%% \else
%%   \usepackage[dvips]{graphicx}
%% \fi


% load tocbibind package 
%   - do not include table of contents in itself
%   - convert the name of bibliography to references
\usepackage[nottoc]{tocbibind}

% load sverb package
%   - enhanced handling of verbatim stuff; listing environment
\usepackage{sverb}

% load listings package
%   - handle inclusion of source code
\usepackage{listings}

% load fancyheaders package
%   - the actual headers and footers are set later
\usepackage{fancyhdr}

% load itpackage 
%   - additional defines and stuff
\usepackage{thesis/itpackage}

% uncomment the following snippet to get rid of Luku/Chapter text at the
% beginning of each Chapter... 
%\makeatletter
%\renewcommand{\@chapapp}{\relax}
%\renewcommand{\@makechapterhead}[1]{%
%  \vspace*{50\p@}%
%  {\parindent \z@ \raggedright \normalfont
%    \ifnum \c@secnumdepth >\m@ne
%        \huge\bfseries \@chapapp\space \thechapter\space\space
%    \fi
%    \interlinepenalty\@M
%    \Huge \bfseries #1\par\nobreak
%    \vskip 40\p@
%  }}
%\makeatother
\begin{document}

% Select language here by removing comment from one of the two rows below
%\selectlanguage{finnish}\fintrue
\selectlanguage{english}\finfalse

% Set some names based on selected language; no modification required
\iffin
\settocbibname{Lähdeluettelo}
\renewcommand{\appname}{Liitteet}
\else
\settocbibname{References}
\renewcommand{\appname}{Appendices}
\fi

% Fill in your information below
\workinfo{Jani Anttonen}
{Proof of Proximity using Verifiable Delay Functions}
{Jaakko Järvi}
{Second Supervisor}
{2020}
{November}
{Kuukausi}

% Set the type of your thesis (Diplomityö, TkK -tutkielma, etc.) and
% laboratory or field of study below
\worktype{Type of thesis}{M.Sc.} 
\deptinfo{Laboratory Name}{Tulevaisuuden teknologioiden laitos}

% Generate the title page 
\gentitle

% Include the abstract
\begin{ittiivis}{tähän, lista, avainsanoista}
Tarkempia ohjeita tiivistelmäsivun laadintaan läytyy opiskelijan
yleisoppaasta, josta alla lyhyt katkelma.

Bibliografisten tietojen jälkeen kirjoitetaan varsinainen tiivistelmä.
Sen on oletettava, että lukijalla on yleiset tiedot aiheesta.
Tiivistelmän tulee olla ymmärrettävissä ilman tarvetta perehtyä koko
tutkielmaan. Se on kirjoitettava täydellisinä virkkeinä,
väliotsakeluettelona. On käytettävä vakiintuneita termejä. Viittauksia
ja lainauksia tiivistelmään ei saa sisällyttää, eikä myäskään tietoja
tai väitteitä, jotka eivät sisälly itse tutkimukseen. Tiivistelmän on
oltava mahdollisimman ytimekäs n. 120 -- 250 sanan pituinen itsenäinen
kokonaisuus, joka mahtuu ykkäsvälillä kirjoitettuna vaivatta
tiivistelmäsivulle. Tiivistelmässä tulisi ilmetä mm.  tutkielman aihe
tutkimuksen kohde, populaatio, alue ja tarkoitus käytetyt
tutkimusmenetelmät (mikäli tutkimus on luonteeltaan teoreettinen ja
tiettyyn kirjalliseen materiaaliin, on mainittava tärkeimmät
lähdeteokset; mikäli on luonteeltaan empiirinen, on mainittava käytetyt
metodit) keskeiset tutkimustulokset tulosten perusteella tehdyt
päätelmät ja toimenpidesuositukset asiasanat
\end{ittiivis}

% if your thesis is in english then this is also required (is it???)
%\begin{itabstract}{list, of, keywords}
%If your thesis is in english this might also be required???
%\end{itabstract}

% empty pagestyle for table of contents etc. 
%
% the redefinition of plain style works also for 1st pages of chapters,
% which is the default in report class. Just state \thispagestyle{empty}
% after \chapter{something} if you want empty style for the 1st pages. 
%
\pagestyle{empty}
\fancypagestyle{plain}{
  \fancyhf{}
  \renewcommand{\headrulewidth}{0 pt}
}

% roman numbering for table of contents etc.
\pagenumbering{roman}

% insert table of contents, list of figures, list of tables
%
% setting the counter to zero effectively removes the page number from
% the toc, lof etc. entries in the toc since there is no roman numeral
% for zero ;-) (if you want them without numbering)
%
%\setcounter{page}{0}
\tableofcontents
\clearpage
%\setcounter{page}{0}
%\listoffigures 
%\clearpage
%\setcounter{page}{0}
%\listoftables

% possibly insert 'list of acronyms'
%
%   - create a chapter called List Of Acronyms (or whatever), which
%     should contain all your acronym definitions, e.g. 
%     \chapter{List Of Acronyms} 
%   - the secnumdepth trickery is needed because acronyms are as a
%     standard chapter and we are faking '\listofacronyms'
%
%\setcounter{secnumdepth}{-1}
%\input{your acronym chapter's file name}
%\setcounter{secnumdepth}{2}

% setup page numbering, page counter, etc.
\startpages

%%%%%%%%%%%%%%%%%%%%%%%%%%%%%%%%%%%%%%%%%%%%%%%%%%%%%%%%%%%%%%%%%%%%%%%%%%%
%
% Thesis starts here. Create a new tex file for each chapter and input it below. You may encounter errors if you use å, ä, ö or <space> characters in referred names.
%
% Good luck!
%
%%%%%%%%%%%%%%%%%%%%%%%%%%%%%%%%%%%%%%%%%%%%%%%%%%%%%%%%%%%%%%%%%%%%%%%%%%%

\chapter{Introduction}
\label{Introduction}

Viittaaminen lukuun \ref{Introduction}, toiseen lukuun \ref{Toinen luku}, alilukuun \ref{Alaotsikko}, tätä alempaan lukuun \ref{Alempiotsikko}, alimpaan lukuun \ref{Alinotsikko}, kuvaan \ref{Kuva esimerkki} ja tauluun \ref{Taulu esimerkki}.

Kuva liitetäään seuraavasti. ShareLaTeXin autocomplete rakentaa koko begin-end blockin yleensä puolestasi.

\begin{figure}
\centering
\includegraphics[width=0.5\textwidth]{kuvat/turun_yliopisto_logo_rgb.png}
\caption{Kuvan otsikko}
\label{Kuva esimerkki}
\end{figure}

Taulukkoja tehdään seuraavasti.

\begin{table}
\centering
\caption{Taulukon otsikko tulee taulun yläpuolelle}
\begin{tabular}{l|c|r|}
 Taulun             &   elementit   & erotetaan \\
 \hline
 toisistaan         &   et-merkillä & \\
 soluja voi myös    &               & jättää tyhjäksi
\end{tabular}
\label{Taulu esimerkki}
\end{table}

%Kirjallisuusviitteet lisätään bib-muodossa bibliografiatiedostoon ja niihin viitataan niiden ID:llä, joka on bib-muodon ensimmäinen kenttä \cite{crawley2007write}.

\section{Background and Motivation}
\label{Background and Motivation}

Distributed systems are vulnerable to problems related to data integrity, because in many cases there is no single source of truth. Work pioneered by the likes of Leslie Lamport has tried to mitigate these problems with clock synchronization and consensus algorithms, which have fixed some but not all of the underlying issues related to the field.

\subsection{Alempiotsikko}
\label{Alempiotsikko}

Lorem ipsum dolor sit amet, consectetur adipiscing elit. Etiam eget tellus porttitor, tempus lacus non, pellentesque ligula. Donec sit amet erat condimentum, feugiat mi accumsan, euismod quam.

Mauris laoreet maximus aliquet. Mauris at gravida elit. Ut nec lobortis elit. Sed lacinia nisi in ex sollicitudin, ac consequat lacus imperdiet. Etiam et velit eu lacus maximus faucibus.

\subsubsection{Alinotsikko, joka ei näy sisällysluettelossa}
\label{Alinotsikko}

Lorem ipsum dolor sit amet, consectetur adipiscing elit. Etiam eget tellus porttitor, tempus lacus non, pellentesque ligula. Donec sit amet erat condimentum, feugiat mi accumsan, euismod quam.

\paragraph{Otsikko tekstissä, joka ei näy sisällysluettelossa}Mauris laoreet maximus aliquet. Mauris at gravida elit. Ut nec lobortis elit. Sed lacinia nisi in ex sollicitudin, ac consequat lacus imperdiet. Etiam et velit eu lacus maximus faucibus. Vestibulum ante ipsum primis in faucibus orci luctus et ultrices posuere cubilia Curae; Donec vulputate tellus ullamcorper odio sodales, non scelerisque neque eleifend. 

\chapter{Toisen luvun otsikko}
\label{Toinen luku}

Lisää tekstiä. Lisää tekstiä. Lisää tekstiä. Lisää tekstiä. Lisää tekstiä. Lisää tekstiä. Lisää tekstiä. Lisää tekstiä. Lisää tekstiä. Lisää tekstiä. Lisää tekstiä. Lisää tekstiä.
%\input{file_name_of_chapter_x}
%\input{file_name_of_chapter_y}


% insert references
%  - included unsrtf.bst provides finnish language version of unsrt
\iffin
\bibliographystyle{thesis/unsrtf}
\else
\bibliographystyle{unsrt}
\fi
\bibliography{library}

%%%%%%%%%%%%%%%%%%%%%%%%%%%%%%%%%%%%%%%%%%%%%%%%%%%%%%%%%%%%%%%%%%%%%%%%%%%
%
% Almost there....
%
%%%%%%%%%%%%%%%%%%%%%%%%%%%%%%%%%%%%%%%%%%%%%%%%%%%%%%%%%%%%%%%%%%%%%%%%%%%

% make sure pagecount is correct even if references overflow to a new page
\pagebreak\addtocounter{page}{-1}
\eofpages
\appendices

% create your appendix chapters with command \appchapter{some name} instead
% of \chapter{some name} for the automagic page counting to work
%\input{file name of appchapter xxx}
%\input{file name of appchapter yyy}
%\input{file name of appchapter zzz}
%\input{and so on}

%%%%%%%%%%%%%%%%%%%%%%%%%%%%%%%%%%%%%%%%%%%%%%%%%%%%%%%%%%%%%%%%%%%%%%%%%%%
%
% main document ends here
%
%%%%%%%%%%%%%%%%%%%%%%%%%%%%%%%%%%%%%%%%%%%%%%%%%%%%%%%%%%%%%%%%%%%%%%%%%%%
\eofapppages
\end{document}
