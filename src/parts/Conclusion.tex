\chapter{Conclusion}
\label{Conclusion}
This thesis presents a protocol for measuring and proving network latency with VDFs between two peers that is meant primarily for optimizing P2P networks. The inclusion of vector commitments is meant to remove needed trust between the participants and third parties that receive and validate the proofs upon new network connections.

Since calculating a VDF is relatively easy for modern processors, a VDF over as little as a few milliseconds of time can be a valid way of measuring latency. Still, Proof of Latency is also a measurement of processing performance. This might introduce an unfortunate barrier for entry for mobile and IoT devices. When used in P2P routing optimization, PoL should result in a network topology organized by a gradient that is defined by geographical location, network connection speed and the similarity in performance between peers.

This means that closely located and similarly performant devices form strong local topologies that are bridged by their random connections to other peers and also by the connections they have to performant local peers. Highly performant devices form strong connections with each other whether they are located near each other or not, because other devices cannot compete against them in the race that is Proof of Latency, resulting in a network topology that is locally effective but global at the same time. The reality that VDFs can't be sped up by additional processors or cores means that the gradient will not be as drastic as one might imagine.

Proof of Latency as a peer scoring metric not only protects peers from eclipse attacks, but can also function as a way of speeding up the initial bootstrapping process by bringing peers more closely together. This also makes the resulting P2P network more robust, due to its locality, in instances of internet stoppages between continents, censorship, or high network load. When used to prove a geographical location, Proof of Latency can combat fraud in applications that rely on GPS location.

\section{Future Considerations}
If the system presented in this thesis was integrated to a blockchain or a publicly verifiable source of randomness for the initial setup, the proofs could be verified by anyone against consensus. Not only this would add trust to the latency measurements, but also speed up initial bootstrapping of the P2P network. When P2P networks eventually grow larger and larger, the network bootstrapping needs to be rethought to handle more traffic, be more decentralized, and be faster in its initialization. By getting introduced to the closest peers possible right at the start the user can experience a more performant network faster, lowering the barrier for entry and making first impressions better.

The fast pace of development of the cryptography field leaves challenges for the security of the protocol. Making sure the protocol is VDF agnostic and quantum resistant would be a logical next step, since the field is still in progress of finding the best possible formulation of a VDF. Quantum computing can render most existing VDF implementations insufficient, and new VDF implementations could change the parameter logic fundamentally.
