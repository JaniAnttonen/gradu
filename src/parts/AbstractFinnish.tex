% !TeX root = ../index.tex
\begin{ittiivis}{todennettavat viivefunktiot, vertaisverkot, vektorisitoumukset, et\"aisyydenmittausprotokollat}
	T\"ass\"a tutkielmassa esit\"an interaktiivisen protokollan nimeltä Proof of latency (PoL), joka pyrkii parantamaan yhteyksi\"a vertaisverkoissa mittaamalla viivett\"a todennettavasta viivefunktiosta rakennetulla loogisella kellolla. Proof of latency koostuu kolmesta algoritmista, \(Setup(e, \lambda)\), \(VCOpen(c, e)\) ja \(Measure(g, T, l_p, l_v)\). Setup luo vektorisitoumuksen, josta luodaan avaus algoritmissa VCOpen avaamalla vektorisitoumus indeksist\"a, joka kuvautuu toisen mittaavan osapuolen julkiseen avaimeen. T\"at\"a avausta k\"aytet\"a\"an luomaan yleinen viitemerkkijono, jota k\"aytet\"a\"an algoritmissa Measure alkupisteen\"a molempien osapuolien todennettavissa viivefunktioissa mittaamaan viivett\"a. Jos kumpikin osapuoli ei huomaa virheit\"a mittauksessa, on heid\"an allekirjoittama todistus valmis mainostettavaksi vertaisverkossa.

PoL ei ota kantaa sen k\"aytt\"amien kryptografisten funktioiden implementaatioon. T\"ast\"a huolimatta olen ohjelmoinut protokollasta prototyypin Rust-ohjelmointikielell\"a k\"aytt\"aen RSA-2048:tta, Catalano-Fiore--vektorisitoumuksia ja Wesolowskin todennettavaa viivefunktiota protokollan esittelyyn. 

Todistettavat viivefunktiot ovat osoittaneet hy\"odyllisiksi aikaleimauksessa, mik\"a n\"aytt\"aisi osoittavan niiden soveltumisen my\"os ajan mittaamiseen t\"ass\"a konteksissa, huolimatta siit\"a ett\"a jokaisen osapuolen tulee ilmoittaa julkisesti teholukema, joka kuvaa niiden tehokkuutta viivefunktioiden laskemisessa. Toinen osapuoli k\"aytt\"a\"a t\"at\"a lukemaa arvioimaan, että valehteliko toinen viivemittauksessa. Olen kuvitellut monta k\"aytt\"okohdetta PoL:lle, kuten maantieteellisen sijainnin todistaminen, suorituskykytestaus, tai itse viivetodistuksien k\"aytt\"aminen uusien viivetodistusten laskemisessa vertaisverkon osallistujien v\"alill\"a.

T\"all\"a hetkell\"a PoL toimii et\"aisyydenmittausprotokollana kahden osallistujan v\"alill\"a, jos niiden suorituskyvyt ovat tarpeeksi l\"ahell\"a toisiaan. Protokolla tarvitsee lis\"atutkimusta sen suhteen, voiko se toimia uskottavana todistuksena kolmansille osapuolille kahden vertaisverkon osallistujan v\"alisest\"a viiveest\"a.
\end{ittiivis}
