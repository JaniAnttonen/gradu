% !TeX root = ../index.tex
\begin{itabstract}{todennettavat viivefunktiot, vertaisverkot, vektorisitoumukset, etäisyydenmittausprotokollat}
	Tässä tutkielmassa esitän interaktiivisen protokollan nimeltä Proof of latency (PoL), joka pyrkii parantamaan yhteyksiä vertaisverkoissa mittaamalla viivettä todennettavalla viivefunktiolla rakennetulla loogisella kellolla. Proof of latency koostuu kolmesta algoritmistä, \(Setup(e, \lambda)\), \(VCOpen(c, e)\) ja \(Measure(g, T, l_p, l_v)\). Setup luo vektorisitoumuksen, josta luodaan avaus algoritmissa VCOpen avaamalla vektorisitoumus indeksistä, joka kuvautuu toisen mittaavan osapuolen julkiseen avaimeen. Tätä avausta käytetään luomaan yleinen viitemerkkijono, jota käytetään algoritmissa Measure alkupisteenä molempien osapuolien todennettavissa viivefunktioissa mittaamaan viivettä. Jos kumpikin osapuoli ei huomaa virheitä mittauksessa, on heidän allekirjoittama todistus valmis mainostettavaksi vertaisverkossa.

PoL ei ota kantaa sen käyttämien kryptografisten funktioiden implementaatioon. Tästä huolimatta olen ohjelmoinut protokollasta prototyypin Rust-ohjelmointikielellä käyttäen RSA-2048:tta, Catalano-Fiore--vektorisitoumuksia ja Wesolowskin todennettavaa viivefunktiota protokollan esittelyyn. 

Todistettavat viivefunktiot ovat osoittaneet hyödyllisiksi aikaleimauksessa, mikä näyttäisi osoittavan niiden soveltumisen myös ajan mittaamiseen tässä konteksissa, huolimatta siitä että jokaisen osapuolen tulee ilmoittaa julkisesti teholukema, joka kuvaa niiden tehokkuutta viivefunktioiden laskemisessa. Toinen osapuoli käyttää tätä lukemaa arvioimaan, että valehteliko toinen viivemittauksessa. Olen kuvitellut monta käyttökohdetta PoL:lle, kuten maantieteellisen sijainnin todistaminen, suorituskykytestaus, tai itse viivetodistuksien käyttäminen uusien viivetodistusten laskemisessa vertaisverkon osallistujien välillä.

Tällä hetkellä PoL toimii etäisyydenmittausprotokollana kahden osallistujan välillä, jos niiden suorituskyvyt ovat tarpeeksi lähellä toisiaan. Protokolla tarvitsee lisätutkimusta sen suhteen, voiko se toimia uskottavana todistuksena kolmansille osapuolille kahden vertaisverkon osallistujan välisestä viiveestä.
\end{itabstract}
