% Tässä metadata gradujen pdf/a tiedostomuotoa varten. Muista päivittää! Tähän ei voi laittaa ääkkösiä
\begin{filecontents*}{\jobname.xmpdata}
  \Title{Gradu}
  \Author{Jani Anttonen}
\end{filecontents*}
  \documentclass{wihuri}
  %\usepackage{isolatin1} % Saadaan ääkköset toimimaan !
  %\usepackage[latin1]{inputenc} % Saadaan oikeat merkit
  \usepackage[utf8]{inputenc} % Tällä toimii utf-8
  \usepackage[T1]{fontenc}      % Ja tämä liittyy edelliseen
  \usepackage[finnish]{babel} %Suomenkielinen tavutus
  \usepackage{tytiivis} %Tiivistelmäsivun laatimiseksi
  \usepackage{graphicx}%Saadaan kuvat toimimaan
  \usepackage{xcolor}
  \usepackage{lastpage}
  \usepackage[a-1b]{pdfx}  % pdf:n tulee graduissa olla pdf/a-1b-standardin mukaista. Tällä ja pdflatexia käyttämällä se onnistuu
  \usepackage[pdfa]{hyperref} % Tämä tarvitaan, jos haluaa sisällysluettelon klikattavaan muotoon. 
  \hypersetup{
      colorlinks = false,
      linkbordercolor = {white}
  }
  % pdflatex vaatii, että kuvat ovat jotain muuta kuin eps-muotoisia. Esimerkiksi pdf käy mainiosti vektorikuville 
  % ja png pikselikuville. Pikselikuvat mennevät paremmin läpi gradujen validoinnista.
  %
  % Esimerkkejä uusien käskyjen määrittelyistä.
  % Käsky \mathbi{``vektorin symboli''} luo boldin italicin kirjaimen. Kreikkalaisille
  % kirjaimille taitaa olla pakko käyttää \pmb:tä.
  \newcommand{\mathbi}[1]{\textbf{\em #1}}
  % Käsky \der luo derivaatan d:n
  \newcommand{\der}{\mbox{d}}
  %
  \begin{document}
  \title{Using Grover's Search Algorithm in Content-Addressed Networks}
  \opinnayte{Pro Gradu}
  \author{Jani Anttonen}
  \vuosi{2019}
  \oppiaine{Tietojenkäsittelytiede}
  \tarkastaja{P.P.}
  \tarkastaja{H.H.}
  \maketitle
  \newpage
  \thispagestyle{empty}
  \vspace*{10cm}
  
  \vfill
  
  \hspace*{-2cm}\parbox{\textwidth}{Turun yliopiston laatujärjestelmän mukaisesti
    tämän julkaisun alkuperäisyys on tarkastettu Turnitin
    OriginalityCheck-järjestelmällä} 
  %Huomaa, että joudut kuitenkin printtaamaan tämän sivun erikseen
  %kaksipuoleseksi kannen kanssa.
  
  
  \newpage
  \begin{tiivistelma}%
          {Fysiikan laitos}%
          {Opiskelija, Olli}%
          {Tutkielman otsikko}
          {Pro Gradu, \pageref{LastPage} s., 3 liites.}%
          {Fysiikka}% Oppiaine
          {Huhtikuu 2004}%
    Tiivistä tähän !
  \end{tiivistelma}
  \tableofcontents %Sisällysluettelo
  \newpage
  \section*{Johdanto}
  \addcontentsline{toc}{section}{Johdanto}
  %Näin tehtynä Johdannolle ei tule numeroa sisälllysluetteloon
  
  Gradua kirjoitettaessa on hyvä muistaa muutamat perussäännöt:
  \begin{enumerate}
  \item Kaikkiin
  kuviin tulee viitata tekstissä, esim. ``Kuvasta \ref{kuva1} nähdään,
  että kuviin viittaaminen on latexissa lastenleikkiä''.
  \item Kuvat ja taulukot kuuluvat oikeasti sivujen ylälaitaan. Latex
    tekee tämän automaattisesti oikein, älä lisäile mitään paikkamääreitä.
  \item Kuvat tulisi laatia kohtuullisen tiiviiksi. Siten, että kuva-ala
    tulee kokonaan hyötykäyttöön.
  \item Kuva- ja taulukkoteksteissä kuuluu olla niin paljon tietoa, että
    kuva/taulukko on ymmärrettävissä ilman tekstin lukua, mm. suureet ja
    lyhenteet tulee esitellä.
  \item Jos otat kuvan jostain lähteestä, muista viitata. Gradut menevät
    myös sähköiseen arkistoon: muista copyright!
  \item Esittele kaikki lyhenteet ensimmäisen käytön yhteydessä:
    esim. elektronimikroskooppi (SEM).
  \item Jos joudut keksimään itse käännöksiä termeille, lisää
    ensimmäisen käyttökerran jälkeen alkuperäinen
    termi. Esim. lukkiutumispotentiaali (engl. pinning potential)
  \item Suureet kirjoitetaan italicilla, kuten $\rho = m/V$. Yksiköt sen
    sijaan romanilla, esim. 1 m$^2$. Vektorit boldilla italicilla,
    $\mathbi{v}$.
  \item Kaavat ovat osa tekstiä, näin ollen pilkut ja pisteet tulevat
    kaavan sisään.
  \item Kaavojen jälkeen esitellään kaikki uudet suureet. Esim Newtonin
    toinen laki on 
  \begin{equation}
  \mathbi{F} = m\mathbi{a},
  \end{equation}
  missä $\mathbi{F}$ on kappaleeseen vaikuttava voima, $m$ on kappaleen
  massa ja $\mathbi{a}$ on sen kiihtyvyys.
  \item Jos koko kappaleen tiedot ovat yhdestä lähteestä, lähdeviite
    tulee kappaleen loppuun, pisteen jälkeen. Kaikissa muissa
    tapauksissa ennen pistettä. Muista viitata aina, kun otat käyttöön 
    numeroarvoja tai muuta tarkkaa tietoa.
  \end{enumerate}
  
  Näitä noudattamalla saadaan vähennettyä ainakin yksi tarkastuskierros.
  
  \section{Tästä alkaa teoriaosuus}
  %esimerkki kuvan liittämisestä
  \begin{figure}
  \begin{center}
  \setlength{\unitlength}{1cm}
  \begin{picture}(6,6)(-3,-3)
  \put(-1.5,0){\vector(1,0){3}}
  \put(2.7,-0.1){$\chi$}
  \put(0,-1.5){\vector(0,1){3}}
  \multiput(-2.5,1)(0.4,0){13}
  {\line(1,0){0.2}}
  \multiput(-2.5,-1)(0.4,0){13}
  {\line(1,0){0.2}}
  \put(0.2,1.4)
  {$\beta=v/c=\tanh\chi$}
  \qbezier(0,0)(0.8853,0.8853)
  (2,0.9640)
  \qbezier(0,0)(-0.8853,-0.8853)
  (-2,-0.9640)
  %\put(-3,-2){\circle*{0.2}}
  \end{picture}
  % Graduissa kannattaa toistaiseksi käyttää png-kuvia, jotka menevät todennäköisemmin läpi pdf/a-validoinnista.
  %\includegraphics[width=10cm]{Pinning_centre.png}
  \caption{Tässä on hieno kuva}
  \label{kuva1}
  \end{center}
  \end{figure}
  
  %\newpage
  % Rivinväli pienemmäksi viiteluettelossa. Fonttia on vaihdettava, jotta käsky
  % toimisi !
  \renewcommand{\baselinestretch}{1}\large\normalsize
  %
  \begin{thebibliography}{50}% Viiteluettelo. TÄTÄ ON PAKKO KÄYTTÄÄ !
  % Jaa, ai miksi ? No, koska tällä tavalla se on vaan niin pirusti
  % helpompaa.
  \bibitem{lshort} T. Oetiker, H. Partl, I. Hyna and E. Schlegl,
  Not so short introduction to \LaTeX 2e, 1998
  \end{thebibliography}
  %
  % Vaihtoehtoisesti thebibliography ympäristölle voi käyttää BibTeX
  % tietokantaa, jonka voit luoda tai käyttää olemassaolevaa (esim.
  % Wihurilla). Suosittelemme tätä lämpimästi!
  %
  % Bibtex-tietokannan saa helposti tehtyä esim TeXMakerilla. Sitten
  % vaan ajetaan latex gradu, bibtex gradu, latex gradu ja latex
  % gradu. Ja TADAA viitteet ovat oikeassa järjestyksessä.
  %
  %\bibliography{/var/bib/yhdistetty}
  %\bibliographystyle{wihuri}
  %
  \end{document} % Se oli siinä !
