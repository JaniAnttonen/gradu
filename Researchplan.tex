\documentclass[a4paper,12pt]{report}

\setlength{\oddsidemargin}{1.85cm} %kaksipuoliset marginaalit
\setlength{\evensidemargin}{0.35cm} %kaksipuoliset marginaalit

% Riviväli 1.3 vastaamaan yleisesti käytettyä 1,5 riviväliä
\linespread{1.3}

% --- Yleiset paketit ---
\usepackage[utf8]{inputenc}
\usepackage[T1]{fontenc}
\usepackage{lmodern}
\usepackage{microtype}
\usepackage{amsfonts,amsmath,amssymb,amsthm,booktabs,color,enumitem,graphicx}
\usepackage[pdftex,hidelinks]{hyperref}
\usepackage[parfill]{parskip}
\usepackage{nameref}
\usepackage{float}
\usepackage{comment}
\usepackage{listings}
\usepackage{hyphenat}

% YLEISET TIEDOT JULKAISUSTA
\title{Research Plan – Proof of Proximity Using a Verifiable Delay Function}
\author{Jani Anttonen, 505933}
\date{\today}

\begin{document}

% --- Front matter ---

\maketitle        % OTSAKESIVU

\section{Introduction}
Johdannossa käsitellään tutkimuksen taustaa ja kuvaillaan tutkimusalueen aihepiiriä
laajemmin. Johdannossa kerrotaan aiheesta tehdystä aikaisemmasta tutkimuksesta. Tätä
viitekehystä vasten kerrotaan, miksi tutkimus on tehty ja mikä merkitys tutkimuksella on.
Johdannossa voidaan myös kertoa, mitä tutkimuksessa tullaan tekemään toisin verrattuna
aikaisempiin tutkimuksiin. 

\section{Goals}
Tutkimuksen tavoitteet kuvataan omassa luvussaan. Luvussa kerrotaan, mikä on
tutkimuksen tavoite – mitä tässä tutkimuksessa tutkitaan ja miksi. Tavoite on hyvä esittää
mahdollisimman selkeästi ja täsmällisesti. Sitä voi lisäksi konkretisoida muutamilla
tutkimuskysymyksillä, joihin tutkimuksessa pyritään saamaan vastaus. 

\section{Materials and methods}
Kuten tutkimuksen tavoitteet, myös tutkimusmenetelmät kuvaillaan erillisessä
nimikoidussa luvussaan. Tutkimusmenetelmät on tutkimussuunnitelman laajin luku,
sisältäen kuvauksen tutkimusaineistosta, sekä suunnitelluista tutkimusaineiston
analysointimenetelmistä. Lisäksi luvussa otetaan kantaa tutkimuksen kannalta merkittäviin
eettisiin kysymyksiin. Eettinen lupa ja sen tarve tulee olla kirjattuna tutkimusmenetelmän
yhteydessä, myös sellaisissa tapauksissa, joissa eettistä lupaa ei tarvita. Mikäli tutkimuksen
tekijä, opiskelija, osallistuu koe-eläintyöhön, tulee tähän tarvittava koulutus ja luvat esittää
tutkimussuunnitelman yhteydessä. Tutkimussuunnitelmassa tulee myös näkyä, mikä on
opiskelijan rooli tutkimuksessa: suunnitelmasta tulee selkeästi ilmetä, mitkä
tutkimusprojektin tehtävät kuuluvat opiskelijalle, mitkä muulle tutkimushenkilökunnalle. 
Tutkimusaineistosta olennaista on kuvailla, kuinka tutkimusaineisto hankitaan, tai on
hankittu. Tärkeää on myös kertoa, millä aikavälillä aineisto kerättiin, tai kuinka kauan sen
kerääminen tulee kestämään. Aineiston analysointimenetelmistä on hyvä raportoida
käytetyt menetelmät sellaisella tarkkuudella, kuin ne ovat tutkimusta suunniteltaessa
tiedossa. Analysointimenetelmän kuvauksen tarkoituksena on kuvailla, millaisin
menetelmin kuvatusta aineistosta pyritään saamaan vastauksia tutkimuskysymyksiin.
Eettisten kysymysten osalta tulee ottaa huomioon tietosuojan ja tietoturvan toteutuminen
sekä tarpeellisten lupien hankintaan kuluva aika. Mikäli tutkimus toteutetaan yhteistyössä
jonkin ulkopuolisen organisaation kanssa, kuten sairaalan, terveyskeskuksen tai koulun,
tulee näiltä tahoilta saada lupa tutkimuksen tekoon. Joidenkin aineistojen käyttöön tarvitaan
lupa Kansaneläkelaitokselta, Väestörekisterikeskukselta tai muilta vastaavilta tahoilta.
Lupien saamiseen on syytä varata aikaa ja varautua mahdollisiin viivästymisiin. 
fsahgjhasjgksha

\section{Timeline}
Toteuttamissuunnitelma sisältää suunnitelman siitä, milloin tutkimustyö aloitetaan tai on 
5
aloitettu ja milloin tutkielma valmistuu. Toteuttamissuunnitelmassa esitellään suuntaaantava arvio eri työvaiheiden kestosta. Aineiston keräys, aineiston analysointi,
kirjallisuuteen perehtyminen ja tutkielman kirjoitusvaihe ovat esimerkkejä tutkimustyöhön
liittyvistä työvaiheista. Toteuttamissuunnitelman tarkkuudeksi riittää kuukausien tasolla
tehty kuvaus eri vaiheista. Toteuttamissuunnitelman tarkoituksena on osoittaa realistinen
suhtautuminen aikatauluun sekä omien resurssien ja tutkielman eri työvaiheiden
tunteminen. 
fsahgjhasjgksha

\end{document}
