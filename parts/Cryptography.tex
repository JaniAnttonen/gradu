% !TeX root = ./Thesis.tex
\chapter{Cryptography}
\label{Cryptography}

% TODO: Nobody can understand crypto based on this definition. Start with an explanation that foresees what the chapter will contain.
Modern cryptography is based on relatively few mathematical fundamentals. The most dominant  have historically been prime numbers and modular multiplication, but elliptic curve cryptography is increasingly considered to be the most secure method out there to keep things colloquial.

Modular multiplication with large numbers has a useful property that you can exchange encrypted data without the participants knowing each other's private keys, but requiring a computation that is theoretically almost impossible to break. Prime numbers and elliptic curves can be used to create the variables used in modular multiplication, providing the basis for the robustness of the cryptosystem.
% HAH TÄÄ ON IHAN PASKA ATM, get a grip

\section{Groups of Unknown Order}

\subsection{RSA}
RSA is named after it's discoverers – Rivest, Shamir, Adleman. It is an asymmetric public-key cryptosystem, meaning that it is based on a keypair, in which one is a public and the other is a secret. The public key is used to encrypt, and the secret key is used to decrypt. This means that everyone who has the public key can encrypt data and that encrypted data is practically impossible to decrypt without the secret key. This works due to the fact that it is hard to factor the product of two large prime numbers.

RSA is an arithmetic trick that creates a mathematical object called a trapdoor permutation. Trapdoor permutation is a function that transforms a number x to a number y in the same range, in a way that computing y from x is easy using the public key but computing x from y is practically impossible without knowing the private key. The private key is the trapdoor.\cite{Aumasson2018-nh}
% TODO: Reader thinks what the hell is the next paragraph about, try to open things up a bit before going into factoring competitions, although they play a big part in VDF history

Some institutions also have been publishing these products of two large prime numbers, claiming they have discarded the two prime numbers that were used in the creation. These are published as public puzzles with prizes for correct factorizations as high as 200000 US dollars, but the larger ones can also be used in cryptography to remove a trusted setup. These products are called RSA numbers, of which RSA-1024 and RSA-2048 are widely used. They can be considered relatively safe for production use, since the latest broken RSA number at the time of writing is RSA-250. Still, in practice, the keys could be still stored by the number's creator, requiring trust towards the institutions or individuals who have published them.

\section{Proofs}
Cryptographic proofs are proofs that depend on the trapdoor nature of cryptographic functions. The most well-known proofs, which are not necessarily thought of as such are cryptographic signatures. Given a public key, a computer in posession of its corresponding private key can produce a signature of any given data, implying together with the data that the computer has seen and processed the data.

Proofs can be private or public. A cryptographic proof can be categorized as public, if a verifier can gather all information required to verify the proof from the transcript of the proof itself, and verify the proof to be correct. Now, since classical cryptography is based on the fact that a computation is asymmetric --- being harder to compute the other way around, a cryptographic proof is still probabilistic in nature, and the security of it based on the parameters used.
