\section{Distance Bounding Protocols}

Distance bounding protocols are interactive protocols that aim to measure the physical distance or latency between the participants. They're latency measurement protocols that are solutions to problems where a simple ping won't suffice, as either party could quite easily cause a false measurement. Distance bounding is used in applications like IP geolocation, RFID and other wireless access control, plus routing in P2P (ad-hoc) networks.

The first distance bounding protocol was designed by Brands and Chaum in 1993 to counter so-called relay attacks.~\cite{Boureanu_undated-bn, Brands1994-hz} A commonly used example of a relay attack is that if the signals between a credit or a debit card and a point of sale system were to be relayed over a distance, two attackers could pay with the relayed info in a totally different country, for example. A distance bounding protocol is safe if information never gets passed faster than the speed of light, and if causality holds due to the prover not able to create a valid response to a challenge before it has received it.~\cite{Boureanu_undated-bn}

The original definition of a distance bounding protocol consisted of three phases: initial phase, critical phase, and a verification phase.~\cite{Brands1994-hz, Mauw2018-uz} During the initial phase the two peers agree on the parameters used. After that, the critical phase is executed, where the two peers do challenge/response rounds. The final phase, the verification phase, is optional, because the verifier can also verify the proofs during the critical phase.

A way to infer physical distance $d$ from the measured round trip time $\Delta t$ is to convert the latency to an approximation of the round trip time $\Delta t$ divided by two times two-thirds the speed of light $c$~\footnote{an approximation of network transmission speed in optic fiber widely used in IP geolocation~\cite{Candela2020-am}}:

\begin{equation*}
  d = \frac{1}{2}\Delta t \frac{2}{3}c
\end{equation*}

Even when not using distance bounding for geolocation one can use the aforementioned method to pick sane parameters for each application, like attack prevention in point of sale systems and RFID lock tags.

\subsection{Attacks}
As distance bounding protocols were originally made to solve the problem of relay attacks, the problem has been mostly solved. Some possible attacks still linger