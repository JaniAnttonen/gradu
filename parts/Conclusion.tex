\chapter{Conclusion}
\label{Conclusion}
Since calculating a VDF is relatively easy for modern processors, a VDF over as little as a few milliseconds of time can be a valid way of measuring latency. Still, without an ASIC chip for calculating VDFs faster than any other available processor, these protocols are also a measurement of processing performance. This might introduce an unfortunate barrier for entry for mobile and IoT devices. The second version of PoL is meant to tackle this problem by creating a performance and latency gradient to the network. The network topology results in a gradient that is defined by geographical location and the similarity in performance. This means that connectedness between mobile and IoT devices is going to be better than between devices that have a huge performance difference.

Proof of Latency as a peer scoring metric not only protects peers from eclipse attacks, but can also function as a way of speeding up the initial botstrapping process by bringing peers more closely together.
