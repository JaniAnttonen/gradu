% !TeX root = ./Thesis.tex
\chapter{Verifiable Delay Functions}
\label{Verifiable Delay Functions}

\section{History}
Verifiable delay functions are based on time-lock puzzles. Time-lock puzzles are computational sequential puzzles that require a certain amount of time to solve.\cite{rivest_time-lock_nodate} Verifiable delay functions fall under the same definition, but introduce a publicly verifiable proof that is much faster to verify than the puzzle was to solve.

\section{Applications}

\section{Variations}


\section{Similar Constructs}
A VDF can only be calculated sequentially, but even without a proof there is a possibility to make the verification faster through parallellism. A non-verifiable delay function, or time-lock puzzle in short, can be still verified faster than the calculation, because there is no sequential requirement after the puzzle has been calculated, enabling to use multiple CPU cores or highly parallel graphics processing units for verifying the puzzle, like in Solana.\cite{yakovenko_solana_2018} 

