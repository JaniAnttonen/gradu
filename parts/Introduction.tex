% !TeX root = ./Thesis.tex
\chapter{Introduction}
\label{Introduction}

Computer applications, whether they are on the public Internet or on a private network, have long preferred a client-server model of communication. In this model, the client application sends requests to the server, and the server responds. Applications are today, however, increasingly moving towards a distributed, peer-to-peer (p2p) networked model, where every peer, whether it is an entire computer or merely a process running on one, serves equally as the both sides of the client-server model. Many uses of p2p do not even get communicated to the end user. For example, the music subscription service Spotify's protocol has been designed to combine server and p2p streaming to improve scalability by decreasing the load on Spotify's servers and bandwith resources.\cite{Kreitz_undated-yp} 

The aspects of of peer-to-peer networking that have recently got attention have been cryptocurrency and blockchain technologies, categorized under the roof term of distributed ledger technology. Peer-to-peer has not been really shown in the public light as anything more than a technology to work around regulation and for doing lawless activities. This is partly because before its use in blockchain applications, it was popularized for its use in file sharing applications like Bittorrent, which it still sees use for. 

% TODO: Reduce the importance of blockchain consensus in the introduction. There are uses in consensus for VDFs, but not as much as other stuff.
Blockchain networks need a way of synchronizing the latest state of the blockchain globally between a number of peers on a p2p network. Since the blockchain data model is sequential and all recorded history is immutable, an algorithm is needed to reach total synchronization between peers. This kind of algorithm is called a consensus algorithm. This problem is not unique to blockchains, it is relevant in distributed databases as well, but public blockchains have raised new issues that have sparked an ongoing development effort for new kinds of consensus algorithms.

Proof of Work is the most used consensus algorithm in public blockchains today, including Bitcoin, of in which whitepaper it was first described in 2008.~\cite{Nakamoto2019-ax} New algorithms have been introduced since to battle Proof of Work's resource intensiveness, including Proof of Stake, which requires network nodes participating in the voting of new blocks to stake a part of their assets as a pawn. Simply this means handing the control of some of the currency owned to the consensus algorithm if the peer wants to participate in the consensus. If a voter gets labeled as malicious, faulty, or absent by a certain majority, it can get slashed, losing all or a part of the staked asset in the process. This serves as an incentive for honest co-operation, with sufficient computation resources.

% TODO: This is not the only, or the first motivation for Verifiable Delay Functions. Correct this.
One problem with Proof of Stake is that the block generation votes are not done globally, but by a selected group of peers called the validators, which vote for the contents of proposed blocks that are generated by just one peer at a time, selected as the block generator. The validators are usually selected randomly. This has generated an increasing demand for verifiable public randomness that is pre-image resistant, meaning the output of the algorithm generating the randomness cannot be influenced before evaluation by input. This created a need for an algorithm that would prevent multiple malicious actors from being selected to vote at once. A cure for this problem is called a verifiable delay function, a VDF in short.

In 2018, two research papers were released independently with similar formalizations of a VDF.~\cite{Wesolowski2018-rf, Pietrzak2018-xs} VDF is an algorithm that requires a specified number of sequential steps to evaluate, but produces a proof that can be efficiently and publicly verified.~\cite{Boneh_undated-ml} To achieve pre-image resistance, a VDF is sequential in nature, and cannot be sped up by parallel processing. There are multiple formulations of a VDF, and not all even have a generated proof, instead using parallel processing with graphics processors to check that the delay function has been calculated correctly.~\cite{Yakovenko2018-zn} This bars less powerful devices, like embedded devices, from verifying the VDF's result efficiently. Thus, generating a proof that requires little time to verify is more ideal.~\cite{Boneh_undated-ml}

% TODO: Elaborate on eclipse attacks, add citations on the matter
Using a verifiable delay function, I propose a novel algorithm for producing a publicly verifiable proof of network latency and difference in computation resources between two participants in a peer-to-peer network. This proof can be used for dynamic routing to reduce latency between peers, and for making eclipse attacks\footnote{Eclipse attack means polluting the target's routing table restricting the target's access to the rest of the network, which opens up other attack possibilities regarding consensus algorithms.} harder to achieve.
