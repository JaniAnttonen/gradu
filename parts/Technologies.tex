% !TeX root = ./Thesis.tex
\chapter{Cryptography}
\label{Cryptography}

\section{RSA}

\section{Asymmetric Cryptography}
\subsection{Diffie-Hellman Key Exchange}


\section{Elliptic Curve Cryptography}

\chapter{Distributed Hash Tables}
\label{DHT}
Distributed hash tables are a way of pointing content to peers in a distributed network. In addition to indexing content in content-addressed networks like IPFS, they can function as routing tables. A hash table is just a regular key-value store, a mapping from a to b. What makes them distributed is the fact that the data stored is meant to be distributed between peers, with not a single peer keeping all the available data in its DHT, but relaying queries that it can't answer to other peers on the network.

\section{Kademlia}
Kademlia is a DHT designed by Petar Maymounkov and David Mazières in 2002.
\subsection{Problem with Randomness}
A single query in Kademlia has been shown in real-world tests to result in an average of 3 network hops, meaning that the query gets relayed through two peers before reaching the requested resource.\cite{roos_comprehending_2013} Network hops are a necessary evil in distributed systems, and Kademlia does well in requiring on average a log(n) queries in a network of n nodes. There's a problem, though. Since the closeness metric is based on a similarity search rather than a measurement, the closest peer is only closest by the identifier, not by network latency.
