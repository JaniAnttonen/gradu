\begin{ittiivis}{tähän, lista, avainsanoista}
Tarkempia ohjeita tiivistelmäsivun laadintaan läytyy opiskelijan
yleisoppaasta, josta alla lyhyt katkelma.

Bibliografisten tietojen jälkeen kirjoitetaan varsinainen tiivistelmä.
Sen on oletettava, että lukijalla on yleiset tiedot aiheesta.
Tiivistelmän tulee olla ymmärrettävissä ilman tarvetta perehtyä koko
tutkielmaan. Se on kirjoitettava täydellisinä virkkeinä,
väliotsakeluettelona. On käytettävä vakiintuneita termejä. Viittauksia
ja lainauksia tiivistelmään ei saa sisällyttää, eikä myäskään tietoja
tai väitteitä, jotka eivät sisälly itse tutkimukseen. Tiivistelmän on
oltava mahdollisimman ytimekäs n. 120 -- 250 sanan pituinen itsenäinen
kokonaisuus, joka mahtuu ykkäsvälillä kirjoitettuna vaivatta
tiivistelmäsivulle. Tiivistelmässä tulisi ilmetä mm.  tutkielman aihe
tutkimuksen kohde, populaatio, alue ja tarkoitus käytetyt
tutkimusmenetelmät (mikäli tutkimus on luonteeltaan teoreettinen ja
tiettyyn kirjalliseen materiaaliin, on mainittava tärkeimmät
lähdeteokset; mikäli on luonteeltaan empiirinen, on mainittava käytetyt
metodit) keskeiset tutkimustulokset tulosten perusteella tehdyt
päätelmät ja toimenpidesuositukset asiasanat
\end{ittiivis}

% if your thesis is in english then this is also required (is it???)
%\begin{itabstract}{list, of, keywords}
%If your thesis is in english this might also be required???
%\end{itabstract}